\documentclass[11pt]{hw-template}

\usepackage{enumitem}
%\usepackage{graphicx}
%\usepackage{mathtools}
%\lstset{showstringspaces=false}
%\setlength\parindent{0pt}

\coursename{Mathematical Ideas and Challenges for Continuous Buhmbox} % DON'T CHANGE THIS

\studname{Alexandre Lamy}
\studmail{all2187@columbia.edu}
\hwNo{}
\collab{}

\begin{document}
\maketitle

\section*{Recall - Normal Buhmbox}
To detect heterogeneity within $D_A$ cases driven by a subgroup that is genetically similar to $D_B$ cases, Buhmbox gets
as input genotype data for $D_A$ cases and controls and looks at $D_B$ SNPs. So we have:
\begin{itemize}
  \item $N$ is the number of cases given.
  \item $N'$ is the number of controls given.
  \item $R$ and $R'$ are the sample correlation matrices of the $D_B$ SNPs for the case and control datasets.
  \item $p_i$ is the risk allele frequency.
  \item $\gamma_i$ is the OR for SNP $i$, i.e. $\gamma_i = \frac{(p_i^+)(1-p_i^+)}{(p_i^-)(1-p_i^-)}$ where $p_i^+$ is RAF in cases and $p_i^-$ is RAF in controls.
\end{itemize}

Then we have 
$$S_{BUHMBOX} = \frac{\sum_{i < j} w_{ij}y_{ij}}{\sqrt{\sum_{i<j}w_{ij}^2}}$$
with
$$Y = \sqrt{\frac{NN'}{N+N'}}(R-R')$$
and
$$w_{ij} = \frac{\sqrt{p_i(1-p_i)p_j(1-p_j)}(\gamma_i - 1)(\gamma_j - 1)}{((\gamma_i-1)p_i + 1)((\gamma_j - 1)p_j + 1)}$$
%%%%%%%%%%%%%%%%%
\section*{Generalization to continuous phenotypes - no cases and controls}
There is a nice extension of the Pearson Correlation Coefficient for weighted observations:

\noindent Define the weighted mean as:
$$m(x, w) = \frac{\sum_i w_ix_i}{\sum_i w_i}$$
The weighted covariance as:
$$cov(x, y, w) = \frac{\sum_i w_i(x_i - m(x, w))(y_i - m(y, w))}{\sum_i w_i}$$
Finally we have that the weighted correlation is given by:
$$corr(x, y, w) = \frac{cov(x, y, w)}{\sqrt{cov(x, x, w)cov(y, y, w)}}$$

However this confines us to using positive weights. But this might make sense as argued
after last meeting (heterogeneity is not symmetric).

Now ideas/challenges on how to replace/change different parts of the Buhmbox equation:
\begin{itemize}
  \item $R- R':$ $R'$ is theoretically the identity matrix (which is irrelevant in the calculation) and
  $R$ could be changed to weighted correlation matrix where weights are given by some increasing function $f$
  on the phenotype. If we do want to get an $R'$, perhaps calculate weighted correlation with flipped weights?
  
  \item $\sqrt{\frac{NN'}{N+N'}}:$ This term is a little weird. Originally in the Buhmbox article they have $Y = \sqrt{N}(R - P)$ where $P$ is the identity (null hypothesis).
  In this case $\sqrt{N}$ is to normalize values in $Y$ to be $N(0, 1)$ since sample correlation has variance $\approx \frac{1 - \rho^2}{N-2}$. $\sqrt{\frac{NN'}{N+N'}}$ is of similar size...
  Perhaps we could just use $\sqrt{N}$ or some sort of function of weights used (even though they are normalized)... Seems very hard to find variance of weighted correlation estimator.
  
  \item $\gamma_i$ could estimate $p_i^+$ by using an increasing weight scheme and $p_i^-$ with a decreasing weighting scheme... But what scheme? In general do we want some sort of exponential, or
  a $\tanh$.
\end{itemize}
%%%%%%%%%%%%%%%%%

\end{document} 
